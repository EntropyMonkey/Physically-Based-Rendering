\section{Report Exercise 4}

Implemented
\begin{itemize}
	\item{an area light using Monte Carlo integration}
	\item{ambient occlusion using cosine weighted sampling}
\end{itemize}

Relevant figures: \ref{fig:comparehardsoftambient}, \ref{fig:comparedirectambient}, \ref{fig:cornellrpp}, \ref{fig:cornellbunnyelephantambientarea}, and \ref{fig:bunnydaytimesambient}. Relevant listings: \ref{lst:arealight::sample}, \ref{lst:ambient::shade}, and \ref{lst:samplecosineweighted}.

 \begin{figure}[h]
	\centering
	\begin{subfigure}[b]{0.4\textwidth}
		\includegraphics[width=\textwidth]{week4/cornellblocks_hardshadows_9rpp_5s.png}
		\caption{Hard shadows, 9 rays per pixel, 5 seconds}
		\label{fig:cornellblockshardshadows}
	\end{subfigure}
	\begin{subfigure}[b]{0.4\textwidth}
		\includegraphics[width=\textwidth]{week4/arealight_9rpp_5s.png}
		\caption{Area light with Monte Carlo integration, 9rpp, 5s}
	\end{subfigure}
	\begin{subfigure}[b]{0.4\textwidth}
		\includegraphics[width=\textwidth]{week4/cornellblocks_ambient_9rpp_20s.png}
		\caption{Area light with Monte Carlo integration and ambient light, 9rpp, 20s, 5 samples}
	\end{subfigure}
	\caption{The Cornellbox with its blocks, comparing hard shadows, soft shadows, and ambient light.}
	\label{fig:comparehardsoftambient}
 \end{figure}
 
 \begin{figure}[h]
	\centering
	\begin{subfigure}[b]{0.5\textwidth}
		\includegraphics[width=\textwidth]{week4/bunny_4rpp_15s.png}
		\caption{The bunny with sunsky lighting model, 4rpp, 15s}
		\label{fig:bunnydirect}
	\end{subfigure}
	\begin{subfigure}[b]{0.5\textwidth}
		\includegraphics[width=\textwidth]{week4/bunny_ambient_9rpp_83s_12.png}
		\caption{The bunny with sunsky lighting model at 12am, with ambient light. 9rpp, 83s}
	\end{subfigure}
	\caption{The bunny in different light conditions. Ambient light adds much to the realism of the picture.}
	\label{fig:comparedirectambient}
 \end{figure}
 
 \begin{figure}[h]
	\centering
	\includegraphics[width=\textwidth]{week4/box_bunny_elephant_ambient_area_9rpp_36s.png}
	\caption{Bunny and elephant inside the cornellbox, ambient and area light. 9rpp, 36s, 5 samples}
	\label{fig:cornellbunnyelephantambientarea}
 \end{figure}
 
 \begin{figure}[h]
	\centering
	\begin{subfigure}[b]{0.4\textwidth}
		\includegraphics[width=\textwidth]{week4/arealight_1rpp_1s.png}
		\caption{1rpp, 1s}
	\end{subfigure}
	\begin{subfigure}[b]{0.4\textwidth}
		\includegraphics[width=\textwidth]{week4/arealight_4rpp_2s.png}
		\caption{4rpp, 2s}
	\end{subfigure}
	\begin{subfigure}[b]{0.4\textwidth}
		\includegraphics[width=\textwidth]{week4/arealight_9rpp_5s.png}
		\caption{9rpp, 5s}
	\end{subfigure}
	\begin{subfigure}[b]{0.4\textwidth}
		\includegraphics[width=\textwidth]{week4/arealight_16rpp_10s.png}
		\caption{16rpp, 10s}
	\end{subfigure}
	\begin{subfigure}[b]{0.4\textwidth}
		\includegraphics[width=\textwidth]{week4/arealight_25rpp_17s.png}
		\caption{25rpp, 17s}
	\end{subfigure}
	\begin{subfigure}[b]{0.4\textwidth}
		\includegraphics[width=\textwidth]{week4/arealight_36rpp_22s.png}
		\caption{36rpp, 22s}
	\end{subfigure}
	\begin{subfigure}[b]{0.4\textwidth}
		\includegraphics[width=\textwidth]{week4/arealight_49rpp_34s.png}
		\caption{49rpp, 34s}
	\end{subfigure}
	
	\caption{Different ray densities, between 1 to 49 rays per pixel.}
	\label{fig:cornellrpp}
 \end{figure}
 
 \begin{figure}[h]
	\centering
	\begin{subfigure}[b]{0.4\textwidth}
		\includegraphics[width=\textwidth]{week4/bunny_ambient_9rpp_86s_7.png}
		\caption{7am, 9rpp, 86s}
	\end{subfigure}
	\begin{subfigure}[b]{0.4\textwidth}
		\includegraphics[width=\textwidth]{week4/bunny_ambient_9rpp_83s_12.png}
		\caption{12am, 9rpp, 83s}
	\end{subfigure}
	\begin{subfigure}[b]{0.4\textwidth}
		\includegraphics[width=\textwidth]{week4/bunny_ambient_9rpp_84s_17.png}
		\caption{7pm, 9rpp, 84s}
	\end{subfigure}
	\begin{subfigure}[b]{0.4\textwidth}
		\includegraphics[width=\textwidth]{week4/bunny_ambient_9rpp_245s_19.png}
		\caption{7pm, 9rpp, 245s}
	\end{subfigure}
	
	\caption{The bunny at different times of the day. Lightmodel: sunsky, using ambient lighting}
	\label{fig:bunnydaytimesambient}
 \end{figure}
 
 \newpage
 \begin{lstlisting}[language=C++,caption=AreaLight::sample,label=lst:arealight::sample,firstnumber=18]
 bool AreaLight::sample(const Vec3f& pos, Vec3f& dir, Vec3f& L) const
{
  // this method uses monte carlo integration to create soft shadows

  // Get geometry info
  const IndexedFaceSet& geometry = mesh->geometry;
	const IndexedFaceSet& normals = mesh->normals;

  // get random triangle
  int triangleIndex = randomizer.mt_random_int32() % geometry.no_faces();
  
  // get index for vertices of triangle
  Vec3i vertexIndex = geometry.face(triangleIndex);
  // get index for normals of triangle
  Vec3i normalIndex = normals.face(triangleIndex);

  // get random position on triangle
  // ref: http://mathworld.wolfram.com/TrianglePointPicking.html
  float sqrt_e1 = sqrt(randomizer.mt_random());
  float e2 = randomizer.mt_random();

  // sample barycentric coordinates
  float u = 1 - sqrt_e1;
  float v = (1 - e2) * sqrt_e1;
  float w = e2 * sqrt_e1;
  
  // linear interpolation of vertices and normals, to get a point on the triangle and the according normal
  Vec3f lightPosition = Vec3f(0.0f);
  Vec3f lightNormal = Vec3f(0.0f);
  
  Vec3f uvw = Vec3f(u, v, w);
  for (int i=0; i<3; i++)
  {
    lightPosition += geometry.vertex(vertexIndex[i]) * uvw[i];
    lightNormal += normals.vertex(normalIndex[i]) * uvw[i];
  }
  lightNormal.normalize();
  
  // get light direction and distance to light
  Vec3f lightDirection = lightPosition - pos;
  float lightDistance = length(lightDirection);

  // set area light direction, normalize
  dir = lightDirection / lightDistance;
  
  // emission is scaled by geometry.no_faces() bec only 1/n are sampled
  Vec3f emission = mesh->face_areas[triangleIndex] * get_emission(triangleIndex) * geometry.no_faces();
  // set radiance
  L = emission * max(dot(lightNormal, -dir), 0.0f) / (lightDistance * lightDistance);

  // trace for shadows
  bool inShadow = false;
  if (shadows)
  {
    Ray shadowRay(pos, dir);
    shadowRay.tmax = lightDistance - 0.1111f;
    inShadow = tracer->trace(shadowRay);
  }

  return !inShadow;
}
 \end{lstlisting}
 
 \newpage
 \begin{lstlisting}[language=C++,caption=Ambient::shade,label=lst:ambient::shade,firstnumber=12]
 Vec3f Ambient::shade(Ray& r, bool emit) const
{ 
  Vec3f rho_d = get_diffuse(r);
  Vec3f radiance(0.0f);

  for (int sample = 0; sample < samples; sample++)
  {
    Ray ray;

    bool inShadow = tracer->trace_cosine_weighted(r, ray);

    if (!inShadow)
    {
      Vec3f sampleRadiance = tracer->get_background(ray.direction);
      radiance += sampleRadiance * dot(r.hit_normal, ray.direction);
    }
  }
  radiance *= rho_d / samples;

  radiance += Lambertian::shade(r, emit);

  return radiance;
}
 \end{lstlisting}
 
 \begin{lstlisting}[language=C++,caption=sampler.h sample\_cosine\_weighted,label=lst:samplecosineweighted]
 inline CGLA::Vec3f sample_cosine_weighted(const CGLA::Vec3f& normal)
{
  // ref: http://www.rorydriscoll.com/2009/01/07/better-sampling/
  // ref: http://pathtracing.wordpress.com/2011/03/03/cosine-weighted-hemisphere/
  // Get random numbers
  const float r1 = randomizer.mt_random();
  const float r2 = randomizer.mt_random();

  const float theta = acos(sqrt(1.0f - r1));
  const float phi = 2.0f * M_PI * r2;

	// Calculate new direction as if the z-axis were the normal
  CGLA::Vec3f _normal(sin(theta) * cos(phi), sin(theta) * sin(phi), cos(theta));

  // Rotate from z-axis to actual normal and return
  rotate_to_normal(normal, _normal);
  return _normal;
}
\end{lstlisting}