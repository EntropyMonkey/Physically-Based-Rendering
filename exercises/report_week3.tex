 \section{Report Exercise 3}
 
 Implemented:
 \begin{itemize}
	\item{Transparent shader}
	\item{Mirror shader}
	\item{Metal shader}
	\item{Russian Roulette}
	\item{fresnel equations for dielectric materials and conductors}
 \end{itemize}
 
 Relevant figures: \ref{fig:spherestransparentmirror}, \ref{fig:transparentelephants}, \ref{fig:transparentelephantsruss}, \ref{fig:bunnyelephant.png}. Relevant listings: \ref{lst:transparent::split_shade}, and \ref{lst:transparent::shade}.
 
 \begin{figure}[h]
	\centering
	\begin{subfigure}[b]{0.4\textwidth}
		\includegraphics[width=\textwidth]{week3/2mirror.png}
		\caption{Two perfect mirror spheres inside of the Cornell Box (using the sunsky lighting).}
		\label{fig:2mirror}
	\end{subfigure}
	\begin{subfigure}[b]{0.4\textwidth}
		\includegraphics[width=\textwidth]{week3/mirrorglass.png}
		\caption{A mirror and a glass sphere (using default lighting).}
	\end{subfigure}
	
	\caption{Two spheres with different shaders.}
	\label{fig:spherestransparentmirror}
 \end{figure}
 
 \begin{figure}[h]
	\centering
	
	\begin{subfigure}[b]{0.3\textwidth}
		\includegraphics[width=\textwidth]{week3/elephant_1_9rpp.png}
		\caption{Transparent elephant with cutoff of 1.}
	\end{subfigure}
	\begin{subfigure}[b]{0.3\textwidth}
		\includegraphics[width=\textwidth]{week3/elephant_2_9rpp.png}
		\caption{Transparent elephant with cutoff of 2.}
	\end{subfigure}
	\begin{subfigure}[b]{0.3\textwidth}
		\includegraphics[width=\textwidth]{week3/elephant_10_9rpp.png}
		\caption{Transparent elephant with cutoff of 10.}
	\end{subfigure}
	
	\caption{Transparent elephant with different cutoff depths. All pictures used 9 rays per pixel.}
	\label{fig:transparentelephants}
 \end{figure}
 
 \begin{figure}[h]
	\centering
	
	\begin{subfigure}[b]{0.3\textwidth}
		\includegraphics[width=\textwidth]{week3/elephant_1_russ.png}
		\caption{Transparent elephant with cutoff of 1, russian roulette.}
	\end{subfigure}
	\begin{subfigure}[b]{0.3\textwidth}
		\includegraphics[width=\textwidth]{week3/elephant_2_russ.png}
		\caption{Transparent elephant with cutoff of 2, russian roulette.}
	\end{subfigure}
	\begin{subfigure}[b]{0.3\textwidth}
		\includegraphics[width=\textwidth]{week3/elephant_10_russ.png}
		\caption{Transparent elephant with cutoff of 10, russian roulette.}
	\end{subfigure}
	
	\caption{Transparent elephant with different cutoff dephts, using russian roulette. 9 rays per pixel}
	\label{fig:transparentelephantsruss}
\end{figure}

\newpage
\begin{lstlisting}[language=C++,caption=Transparent::shade,label=lst:transparent::shade]
Vec3f Transparent::shade(Ray& r, bool emit) const
{  
  Vec3f radiance = Vec3f(0.0f);

  if (r.trace_depth < splits)
  {
    radiance = split_shade(r, emit);
  }
  else if (r.trace_depth < max_depth)
  {
    // refraction
    Ray refracted;
    double fresnelR;
    tracer->trace_refracted(r, refracted, fresnelR); // fresnelR => use as step probability

    // russian roulette for reflections
    float rand = randomizer.mt_random();

    // 1st cond. -> russian roulette with fresnelR => pdf, 2nd cond. -> eliminating rays following surface
    if (rand <= fresnelR && fresnelR > 0.001)
    {
      // reflect
      Ray reflected;
      tracer->trace_reflected(r, reflected);
      radiance = shade_new_ray(reflected); // * fresnelR / fresnelR; // divide by fresnelR, since fresnelR is used as the step probability
    }
    // if not reflecting, take refraction
    else if (1 - fresnelR > 0.001)
    {
      radiance = shade_new_ray(refracted); // * (1 - fresnelR) / (1 - fresnelR);
    }
  }

  return radiance;
}
\end{lstlisting}

\newpage
\begin{lstlisting}[language=C++,caption=Transparent::split\_shade,label=lst:transparent::split_shade]
Vec3f Transparent::split_shade(Ray& r, bool emit) const
{
  Vec3f radiance(0.0f);

  if (r.trace_depth < splits)
  {
    Ray refracted;
    double fresnelR;
    tracer->trace_refracted(r, refracted, fresnelR);

    if (1 - fresnelR > 0.001)
      radiance += shade_new_ray(refracted) * (1.0f - fresnelR);

    // eliminate rays following the surface
    if (fresnelR > 0.001)
    {
      Ray reflected;
      tracer->trace_reflected(r, reflected);
      radiance += shade_new_ray(reflected) * fresnelR;
    }
  }

  return radiance;
}
\end{lstlisting}

\begin{figure}[h]
	\centering
	
	\includegraphics[width=\textwidth]{week3/bunny_el_9rpp.png}
	
	\caption{A silver bunny and a transparent elephant, using russian roulette, 9 rays per pixel.}
	\label{fig:bunnyelephant.png}
\end{figure}