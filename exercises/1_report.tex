\documentclass[a4paper]{article}

\usepackage{graphicx}
\usepackage{caption}
\usepackage{subcaption}
\usepackage{listings}
\usepackage{color}
\usepackage[pdfborder={0 0 0}]{hyperref}

% set listing style
\definecolor{mygreen}{rgb}{0,0.6,0}
\definecolor{mygray}{rgb}{0.5,0.5,0.5}
\definecolor{mymauve}{rgb}{0.58,0,0.82}

\lstset{ %
   backgroundcolor=\color{white},   % choose the background color; you must add \usepackage{color} or \usepackage{xcolor}
   basicstyle=\footnotesize,        % the size of the fonts that are used for the code
   breakatwhitespace=false,         % sets if automatic breaks should only happen at whitespace
   breaklines=true,                 % sets automatic line breaking
   captionpos=t,                    % sets the caption-position to bottom
   commentstyle=\color{mygreen},    % comment style
   deletekeywords={...},            % if you want to delete keywords from the given language
   extendedchars=true,              % lets you use non-ASCII characters; for 8-bits encodings only, does not work with UTF-8
   frame=false,                    	% adds a frame around the code
   keywordstyle=\color{blue},       % keyword style
   language=Octave,                 % the language of the code
   morekeywords={*,...},            % if you want to add more keywords to the set
   numbers=left,                    % where to put the line-numbers; possible values are (none, left, right)
   numbersep=5pt,                   % how far the line-numbers are from the code
   numberstyle=\tiny\color{mygray}, % the style that is used for the line-numbers
   rulecolor=\color{mygray},         % if not set, the frame-color may be changed on line-breaks within not-black text (e.g. comments (green here))
   showspaces=false,                % show spaces everywhere adding particular underscores; it overrides 'showstringspaces'
   showstringspaces=false,          % underline spaces within strings only
   showtabs=false,                  % show tabs within strings adding particular underscores
   stepnumber=2,                    % the step between two line-numbers. If it's 1, each line will be numbered
   stringstyle=\color{mymauve},     % string literal style
   tabsize=2,                       % sets default tabsize to 2 spaces
   title=\lstname                   % show the filename of files included with \lstinputlisting; also try caption instead of title
}

% nice listing captions
\DeclareCaptionFont{white}{ \color{white} }
\DeclareCaptionFormat{listing}{
   \colorbox[cmyk]{0.43, 0.35, 0.35,0.01 }{
     \parbox{\textwidth}{\hspace{15pt}#1#2#3}
   }
}
\captionsetup[lstlisting]{ format=listing, labelfont=white, textfont=white, singlelinecheck=false, margin=0pt, font={bf,footnotesize} }

% don't show section numbers
\renewcommand{\thesection}{}

\begin{document}

% title
\begin{titlepage}
\textsc{\LARGE Physically Based Rendering, Reports}\\[1.5cm]

\emph{Student:}\\
Andrea \textsc{Distler}, 130269\\[1.5cm]

\emph{Teacher:}\\
Jeppe \textsc{Frisvald}\\

\vfill

{\large \today}
\end{titlepage}

% table of contents
\tableofcontents

% contents

\paragraph{The code} can be found on github: \\ \url{https://github.com/JungleJinn/Physically-Based-Rendering}

% ---------------------------------------------------- Exercise 1
\newpage
 \section{Report Exercise 1}

\begin{itemize}
\item{Implemented a directional light with shadows}
\item{Implemented an area light with shadows}
\end{itemize}

Relevant pictures: figures \ref{fig:shadows}, and \ref{fig:arealight}.
Relevant listings: \ref{lst:directional}, \ref{lst:lambertian}, and \ref{lst:arealight}.

\begin{figure}
	\centering
	\begin{subfigure}[b]{0.4\textwidth}
		\includegraphics[width=\textwidth]{week1/bunny_noshadows.png}
		\caption{Without shadows: ~5s}
		\label{fig:bunnynoshadows}
	\end{subfigure}
	\begin{subfigure}[b]{0.4\textwidth}
		\includegraphics[width=\textwidth]{week1/bunny_shadowed.png}
		\caption{With shadows: ~8s}
		\label{fig:bunnyshadowed}
	\end{subfigure}
	\caption{Bunny.obj, Tris: 69451, 36 samples, 1 directional light, Lambertian shader}
	\label{fig:shadows}
\end{figure}

\begin{figure}
	\centering
	\begin{subfigure}[b]{0.4\textwidth}
		\includegraphics[width=\textwidth]{week1/arealight_noshadows.png}
		\caption{Without shadows: 0.3s}
		\label{fig:arealightnoshadows}
	\end{subfigure}
	\begin{subfigure}[b]{0.4\textwidth}
		\includegraphics[width=\textwidth]{week1/arealight_shadows.png}
		\caption{With shadows: 0.5s}
		\label{fig:arealightshadows}
	\end{subfigure}
	\caption{Cornellbox.obj and CornellBlocks.obj, Tris: 36, 4 samples, 1 area light, Lambertian shaders}
	\label{fig:arealight}
\end{figure}

\newpage
\begin{lstlisting}[language=C++,caption=Directional.cpp,label=lst:directional,firstnumber=15]
bool Directional::sample(const Vec3f& pos, Vec3f& dir, Vec3f& L) const
{
  dir = -light_dir;
  L = emission;

  // test for shadow
  Ray shadowRay(pos, -light_dir);
  bool inShadow = false;

  if (shadows)
    inShadow = tracer->trace(shadowRay);

  return !inShadow;
}
\end{lstlisting}

\begin{lstlisting}[language=C++,caption=Lambertian.cpp,label=lst:lambertian,firstnumber=16]
Vec3f Lambertian::shade(Ray& r, bool emit) const
{
  Vec3f rho_d = get_diffuse(r);
  Vec3f result(0.0f);
  
  // temp light direction and radiance
  Vec3f lightDirection, radiance;
  for (std::vector<Light*>::const_iterator it = lights.begin(); it != lights.end(); it++)
  {
    if ((*it)->sample(r.hit_pos, lightDirection, radiance))
    {
      // output of Lambertian BRDF
      Vec3f f = rho_d * M_1_PIf;
      
      // directional light radiance
      // f - scattered light radiance, radiance - current light radiance, last term: cosine cut off at 0
      result += f * radiance * std::max(dot(r.hit_normal, lightDirection), 0.0f);
    }
  }

  return result + Emission::shade(r, emit);
}
\end{lstlisting}

\begin{lstlisting}[language=C++,caption=AreaLight.cpp,label=lst:arealight, firstnumber=18]
bool AreaLight::sample(const Vec3f& pos, Vec3f& dir, Vec3f& L) const
{  
  // Get geometry info
  const IndexedFaceSet& geometry = mesh->geometry;
	const IndexedFaceSet& normals = mesh->normals;

  // averaged light position
  Vec3f lightPosition = Vec3f(0.0f);
  // averaged normals
  Vec3f lightNormal = Vec3f(0.0f);
  // emission summed up from all faces
  Vec3f emission = Vec3f(0.0f);

  // iterate over all faces
  for (int i = 0; i < geometry.no_faces(); i++)
  {
    // get the center of the face
    Vec3i face = geometry.face(i);
    Vec3f v0 = geometry.vertex(face[0]);
    Vec3f v1 = geometry.vertex(face[1]);
    Vec3f v2 = geometry.vertex(face[2]);
    Vec3f faceCenter = v0 + (v1 - v0 + v2 - v0) * 0.5f;
    
    // combine light position
    lightPosition += faceCenter;

    // average normals
    lightNormal += (normals.vertex(face[0]) + normals.vertex(face[1]) + normals.vertex(face[2])) / 3;

    // add emission
    emission += mesh->face_areas[i] * get_emission(i);
  }

  // average light position
  lightPosition /= geometry.no_faces();

  lightNormal.normalize();
  
  // get light direction and distance to light
  Vec3f lightDirection = lightPosition - pos;
  float lightDistance = length(lightDirection);

  // set area light direction, normalize
  dir = lightDirection / lightDistance;

  // set radiance
  L = emission * std::max(dot(-dir, lightNormal), 0.0f) / (lightDistance * lightDistance);

  // trace for shadows
  bool inShadow = false;
  if (shadows)
  {
    Ray shadowRay(pos, dir);
    shadowRay.tmax = lightDistance - 0.1111f;
    inShadow = tracer->trace(shadowRay);
  }

  return !inShadow;
}
\end{lstlisting}

% ---------------------------------------------------- Exercise 2
\newpage
 \section{Report Exercise 2}

The input parameters for the sun sky light are theta, and phi, together creating the solar position, and the turbidity (how much light is scattered due to dirt in the atmosphere, using empirical values - Preetham). Using Preetham's paper, theta and phi are calculated from the latitude, declination, julian day of the year and the time of the day, as well as some constants from Preetham's paper. The code for this can be found in listing \ref{lst:renderengine::init_tracer}.

The model used for calculating the sun's intensity is calculated as shown in listing \ref{lst:preethamsunsky::sample}. The sun covers a solid angle of 2$\pi$ degrees$^2$ ($\Rightarrow$ the whole hemisphere).
\\
\\
Relevant figure: \ref{fig:sunlight}. Relevant listings: \ref{lst:renderengine::init_tracer}, and \ref{lst:preethamsunsky::sample}.

\begin{figure}[h]
	\centering
	\begin{subfigure}[b]{0.3\textwidth}
		\includegraphics[width=\textwidth]{week2/2/bunny_7.png}
		\caption{7am}
		\label{fig:skylight7}
	\end{subfigure}
	\begin{subfigure}[b]{0.3\textwidth}
		\includegraphics[width=\textwidth]{week2/2/bunny_9.png}
		\caption{9am}
		\label{fig:skylight9}
	\end{subfigure}
	\begin{subfigure}[b]{0.3\textwidth}
		\includegraphics[width=\textwidth]{week2/2/bunny_11.png}
		\caption{11am}
		\label{fig:skylight11}
	\end{subfigure}
	\begin{subfigure}[b]{0.3\textwidth}
		\includegraphics[width=\textwidth]{week2/2/bunny_13.png}
		\caption{1pm}
		\label{fig:skylight13}
	\end{subfigure}
	\begin{subfigure}[b]{0.3\textwidth}
		\includegraphics[width=\textwidth]{week2/2/bunny_15.png}
		\caption{3pm}
		\label{fig:skylight15}
	\end{subfigure}
	\begin{subfigure}[b]{0.3\textwidth}
		\includegraphics[width=\textwidth]{week2/2/bunny_17.png}
		\caption{5pm}
		\label{fig:skylight17}
	\end{subfigure}
	\begin{subfigure}[b]{0.3\textwidth}
		\includegraphics[width=\textwidth]{week2/2/bunny_19.png}
		\caption{9pm}
		\label{fig:skylight19}
	\end{subfigure}
	\caption{Bunny.obj and plane, Tris: 70.000, 1 sample, 1 skylight, Lambertian shaders $\Rightarrow$ approx. 3s per picture}
	\label{fig:sunlight}
\end{figure}

\newpage
\begin{lstlisting}[language=C++,caption=RenderEngine::init\_tracer(),label=lst:renderengine::init_tracer]
if(use_sun_and_sky)
  {
    // Use the Julian date (day_of_year), the solar time (time_of_day), the latitude (latitude),
    // and the angle with South (angle_with_south) to find the direction toward the sun (sun_dir).

    // hard coded numbers are from Preetham et al.'s A Practical Analytical Model for Daylight, SIGGRAPH 1999
    float declination = 0.4093 * sin(2 * M_PIf * (day_of_year - 81) / 368);
    float theta = M_PIf * 0.5f - asin(sin(latitude) * sin(declination) - 
      cos(latitude) * cos(declination) * cos(M_PIf * time_of_day / 12));
    float phi = atan(-(cos(declination) * sin(M_PIf * time_of_day / 12)) / 
      (cos(latitude) * sin(declination) - sin(latitude) * cos(declination) * cos(M_PIf * time_of_day / 12)));

    sun_sky.setSunTheta(theta);
    sun_sky.setSunPhi(phi);
    sun_sky.setTurbidity(turbidity);
    sun_sky.init();
    tracer.set_background(&sun_sky);
  }
\end{lstlisting}

\begin{lstlisting}[language=C++,caption=PreethamSunSky::sample(..),label=lst:preethamsunsky::sample]
bool PreethamSunSky::sample(const Vec3f& pos, Vec3f& dir, Vec3f& L) const
{
  dir = const_cast<PreethamSunSky*>(this)->getSunDir();

  float area = 1;
  float solid_angle = 2 * M_PI;
  float cos_theta = dot(Vec3f(0, 1, 0), dir);

  // * 0.00001f to convert to the right unit (cd/m^2)
  L = const_cast<PreethamSunSky*>(this)->sunColor() / (area * solid_angle * cos_theta) * 0.00001f;

  // test for shadow
  Ray shadowRay(pos, dir);
  bool inShadow = false;

  if (shadows)
    inShadow = tracer->trace(shadowRay);

  return !inShadow;
}
\end{lstlisting}

% ---------------------------------------------------- Exercise 3
\newpage
 \section{Report Exercise 3}
 
 Implemented:
 \begin{itemize}
	\item{Transparent shader}
	\item{Mirror shader}
	\item{Metal shader}
	\item{Russian Roulette}
	\item{fresnel equations for dielectric materials and conductors}
 \end{itemize}
 
 Relevant figures: \ref{fig:spherestransparentmirror}, \ref{fig:transparentelephants}, \ref{fig:transparentelephantsruss}, \ref{fig:bunnyelephant.png}. Relevant listings: \ref{lst:transparent::split_shade}, and \ref{lst:transparent::shade}.
 
 \begin{figure}[h]
	\centering
	\begin{subfigure}[b]{0.4\textwidth}
		\includegraphics[width=\textwidth]{week3/2mirror.png}
		\caption{Two perfect mirror spheres inside of the Cornell Box (using the sunsky lighting).}
		\label{fig:2mirror}
	\end{subfigure}
	\begin{subfigure}[b]{0.4\textwidth}
		\includegraphics[width=\textwidth]{week3/mirrorglass.png}
		\caption{A mirror and a glass sphere (using default lighting).}
	\end{subfigure}
	
	\caption{Two spheres with different shaders.}
	\label{fig:spherestransparentmirror}
 \end{figure}
 
 \begin{figure}[h]
	\centering
	
	\begin{subfigure}[b]{0.3\textwidth}
		\includegraphics[width=\textwidth]{week3/elephant_1_9rpp.png}
		\caption{Transparent elephant with cutoff of 1.}
	\end{subfigure}
	\begin{subfigure}[b]{0.3\textwidth}
		\includegraphics[width=\textwidth]{week3/elephant_2_9rpp.png}
		\caption{Transparent elephant with cutoff of 2.}
	\end{subfigure}
	\begin{subfigure}[b]{0.3\textwidth}
		\includegraphics[width=\textwidth]{week3/elephant_10_9rpp.png}
		\caption{Transparent elephant with cutoff of 10.}
	\end{subfigure}
	
	\caption{Transparent elephant with different cutoff depths. All pictures used 9 rays per pixel.}
	\label{fig:transparentelephants}
 \end{figure}
 
 \begin{figure}[h]
	\centering
	
	\begin{subfigure}[b]{0.3\textwidth}
		\includegraphics[width=\textwidth]{week3/elephant_1_russ.png}
		\caption{Transparent elephant with cutoff of 1, russian roulette.}
	\end{subfigure}
	\begin{subfigure}[b]{0.3\textwidth}
		\includegraphics[width=\textwidth]{week3/elephant_2_russ.png}
		\caption{Transparent elephant with cutoff of 2, russian roulette.}
	\end{subfigure}
	\begin{subfigure}[b]{0.3\textwidth}
		\includegraphics[width=\textwidth]{week3/elephant_10_russ.png}
		\caption{Transparent elephant with cutoff of 10, russian roulette.}
	\end{subfigure}
	
	\caption{Transparent elephant with different cutoff dephts, using russian roulette. 9 rays per pixel}
	\label{fig:transparentelephantsruss}
\end{figure}

\newpage
\begin{lstlisting}[language=C++,caption=Transparent::shade,label=lst:transparent::shade]
Vec3f Transparent::shade(Ray& r, bool emit) const
{  
  Vec3f radiance = Vec3f(0.0f);

  if (r.trace_depth < splits)
  {
    radiance = split_shade(r, emit);
  }
  else if (r.trace_depth < max_depth)
  {
    // refraction
    Ray refracted;
    double fresnelR;
    tracer->trace_refracted(r, refracted, fresnelR); // fresnelR => use as step probability

    // russian roulette for reflections
    float rand = randomizer.mt_random();

    // 1st cond. -> russian roulette with fresnelR => pdf, 2nd cond. -> eliminating rays following surface
    if (rand <= fresnelR && fresnelR > 0.001)
    {
      // reflect
      Ray reflected;
      tracer->trace_reflected(r, reflected);
      radiance = shade_new_ray(reflected); // * fresnelR / fresnelR; // divide by fresnelR, since fresnelR is used as the step probability
    }
    // if not reflecting, take refraction
    else if (1 - fresnelR > 0.001)
    {
      radiance = shade_new_ray(refracted); // * (1 - fresnelR) / (1 - fresnelR);
    }
  }

  return radiance;
}
\end{lstlisting}

\newpage
\begin{lstlisting}[language=C++,caption=Transparent::split\_shade,label=lst:transparent::split_shade]
Vec3f Transparent::split_shade(Ray& r, bool emit) const
{
  Vec3f radiance(0.0f);

  if (r.trace_depth < splits)
  {
    Ray refracted;
    double fresnelR;
    tracer->trace_refracted(r, refracted, fresnelR);

    if (1 - fresnelR > 0.001)
      radiance += shade_new_ray(refracted) * (1.0f - fresnelR);

    // eliminate rays following the surface
    if (fresnelR > 0.001)
    {
      Ray reflected;
      tracer->trace_reflected(r, reflected);
      radiance += shade_new_ray(reflected) * fresnelR;
    }
  }

  return radiance;
}
\end{lstlisting}

\begin{figure}[h]
	\centering
	
	\includegraphics[width=\textwidth]{week3/bunny_el_9rpp.png}
	
	\caption{A silver bunny and a transparent elephant, using russian roulette, 9 rays per pixel.}
	\label{fig:bunnyelephant.png}
\end{figure}

% ---------------------------------------------------- Exercise 4
\newpage
\section{Report Exercise 4}

Implemented
\begin{itemize}
	\item{an area light using Monte Carlo integration}
	\item{ambient occlusion using cosine weighted sampling}
\end{itemize}

Relevant figures: \ref{fig:comparehardsoftambient}, \ref{fig:comparedirectambient}, \ref{fig:cornellrpp}, \ref{fig:cornellbunnyelephantambientarea}, and \ref{fig:bunnydaytimesambient}. Relevant listings: \ref{lst:arealight::sample}, \ref{lst:ambient::shade}, and \ref{lst:samplecosineweighted}.

 \begin{figure}[h]
	\centering
	\begin{subfigure}[b]{0.4\textwidth}
		\includegraphics[width=\textwidth]{week4/cornellblocks_hardshadows_9rpp_5s.png}
		\caption{Hard shadows, 9 rays per pixel, 5 seconds}
		\label{fig:cornellblockshardshadows}
	\end{subfigure}
	\begin{subfigure}[b]{0.4\textwidth}
		\includegraphics[width=\textwidth]{week4/arealight_9rpp_5s.png}
		\caption{Area light with Monte Carlo integration, 9rpp, 5s}
	\end{subfigure}
	\begin{subfigure}[b]{0.4\textwidth}
		\includegraphics[width=\textwidth]{week4/cornellblocks_ambient_9rpp_20s.png}
		\caption{Area light with Monte Carlo integration and ambient light, 9rpp, 20s, 5 samples}
	\end{subfigure}
	\caption{The Cornellbox with its blocks, comparing hard shadows, soft shadows, and ambient light.}
	\label{fig:comparehardsoftambient}
 \end{figure}
 
 \begin{figure}[h]
	\centering
	\begin{subfigure}[b]{0.5\textwidth}
		\includegraphics[width=\textwidth]{week4/bunny_4rpp_15s.png}
		\caption{The bunny with sunsky lighting model, 4rpp, 15s}
		\label{fig:bunnydirect}
	\end{subfigure}
	\begin{subfigure}[b]{0.5\textwidth}
		\includegraphics[width=\textwidth]{week4/bunny_ambient_9rpp_83s_12.png}
		\caption{The bunny with sunsky lighting model at 12am, with ambient light. 9rpp, 83s}
	\end{subfigure}
	\caption{The bunny in different light conditions. Ambient light adds much to the realism of the picture.}
	\label{fig:comparedirectambient}
 \end{figure}
 
 \begin{figure}[h]
	\centering
	\includegraphics[width=\textwidth]{week4/box_bunny_elephant_ambient_area_9rpp_36s.png}
	\caption{Bunny and elephant inside the cornellbox, ambient and area light. 9rpp, 36s, 5 samples}
	\label{fig:cornellbunnyelephantambientarea}
 \end{figure}
 
 \begin{figure}[h]
	\centering
	\begin{subfigure}[b]{0.4\textwidth}
		\includegraphics[width=\textwidth]{week4/arealight_1rpp_1s.png}
		\caption{1rpp, 1s}
	\end{subfigure}
	\begin{subfigure}[b]{0.4\textwidth}
		\includegraphics[width=\textwidth]{week4/arealight_4rpp_2s.png}
		\caption{4rpp, 2s}
	\end{subfigure}
	\begin{subfigure}[b]{0.4\textwidth}
		\includegraphics[width=\textwidth]{week4/arealight_9rpp_5s.png}
		\caption{9rpp, 5s}
	\end{subfigure}
	\begin{subfigure}[b]{0.4\textwidth}
		\includegraphics[width=\textwidth]{week4/arealight_16rpp_10s.png}
		\caption{16rpp, 10s}
	\end{subfigure}
	\begin{subfigure}[b]{0.4\textwidth}
		\includegraphics[width=\textwidth]{week4/arealight_25rpp_17s.png}
		\caption{25rpp, 17s}
	\end{subfigure}
	\begin{subfigure}[b]{0.4\textwidth}
		\includegraphics[width=\textwidth]{week4/arealight_36rpp_22s.png}
		\caption{36rpp, 22s}
	\end{subfigure}
	\begin{subfigure}[b]{0.4\textwidth}
		\includegraphics[width=\textwidth]{week4/arealight_49rpp_34s.png}
		\caption{49rpp, 34s}
	\end{subfigure}
	
	\caption{Different ray densities, between 1 to 49 rays per pixel.}
	\label{fig:cornellrpp}
 \end{figure}
 
 \begin{figure}[h]
	\centering
	\begin{subfigure}[b]{0.4\textwidth}
		\includegraphics[width=\textwidth]{week4/bunny_ambient_9rpp_86s_7.png}
		\caption{7am, 9rpp, 86s}
	\end{subfigure}
	\begin{subfigure}[b]{0.4\textwidth}
		\includegraphics[width=\textwidth]{week4/bunny_ambient_9rpp_83s_12.png}
		\caption{12am, 9rpp, 83s}
	\end{subfigure}
	\begin{subfigure}[b]{0.4\textwidth}
		\includegraphics[width=\textwidth]{week4/bunny_ambient_9rpp_84s_17.png}
		\caption{7pm, 9rpp, 84s}
	\end{subfigure}
	\begin{subfigure}[b]{0.4\textwidth}
		\includegraphics[width=\textwidth]{week4/bunny_ambient_9rpp_245s_19.png}
		\caption{7pm, 9rpp, 245s}
	\end{subfigure}
	
	\caption{The bunny at different times of the day. Lightmodel: sunsky, using ambient lighting}
	\label{fig:bunnydaytimesambient}
 \end{figure}
 
 \newpage
 \begin{lstlisting}[language=C++,caption=AreaLight::sample,label=lst:arealight::sample,firstnumber=18]
 bool AreaLight::sample(const Vec3f& pos, Vec3f& dir, Vec3f& L) const
{
  // this method uses monte carlo integration to create soft shadows

  // Get geometry info
  const IndexedFaceSet& geometry = mesh->geometry;
	const IndexedFaceSet& normals = mesh->normals;

  // get random triangle
  int triangleIndex = randomizer.mt_random_int32() % geometry.no_faces();
  
  // get index for vertices of triangle
  Vec3i vertexIndex = geometry.face(triangleIndex);
  // get index for normals of triangle
  Vec3i normalIndex = normals.face(triangleIndex);

  // get random position on triangle
  // ref: http://mathworld.wolfram.com/TrianglePointPicking.html
  float sqrt_e1 = sqrt(randomizer.mt_random());
  float e2 = randomizer.mt_random();

  // sample barycentric coordinates
  float u = 1 - sqrt_e1;
  float v = (1 - e2) * sqrt_e1;
  float w = e2 * sqrt_e1;
  
  // linear interpolation of vertices and normals, to get a point on the triangle and the according normal
  Vec3f lightPosition = Vec3f(0.0f);
  Vec3f lightNormal = Vec3f(0.0f);
  
  Vec3f uvw = Vec3f(u, v, w);
  for (int i=0; i<3; i++)
  {
    lightPosition += geometry.vertex(vertexIndex[i]) * uvw[i];
    lightNormal += normals.vertex(normalIndex[i]) * uvw[i];
  }
  lightNormal.normalize();
  
  // get light direction and distance to light
  Vec3f lightDirection = lightPosition - pos;
  float lightDistance = length(lightDirection);

  // set area light direction, normalize
  dir = lightDirection / lightDistance;
  
  // emission is scaled by geometry.no_faces() bec only 1/n are sampled
  Vec3f emission = mesh->face_areas[triangleIndex] * get_emission(triangleIndex) * geometry.no_faces();
  // set radiance
  L = emission * max(dot(lightNormal, -dir), 0.0f) / (lightDistance * lightDistance);

  // trace for shadows
  bool inShadow = false;
  if (shadows)
  {
    Ray shadowRay(pos, dir);
    shadowRay.tmax = lightDistance - 0.1111f;
    inShadow = tracer->trace(shadowRay);
  }

  return !inShadow;
}
 \end{lstlisting}
 
 \newpage
 \begin{lstlisting}[language=C++,caption=Ambient::shade,label=lst:ambient::shade,firstnumber=12]
 Vec3f Ambient::shade(Ray& r, bool emit) const
{ 
  Vec3f rho_d = get_diffuse(r);
  Vec3f radiance(0.0f);

  for (int sample = 0; sample < samples; sample++)
  {
    Ray ray;

    bool inShadow = tracer->trace_cosine_weighted(r, ray);

    if (!inShadow)
    {
      Vec3f sampleRadiance = tracer->get_background(ray.direction);
      radiance += sampleRadiance * dot(r.hit_normal, ray.direction);
    }
  }
  radiance *= rho_d / samples;

  radiance += Lambertian::shade(r, emit);

  return radiance;
}
 \end{lstlisting}
 
 \begin{lstlisting}[language=C++,caption=sampler.h sample\_cosine\_weighted,label=lst:samplecosineweighted]
 inline CGLA::Vec3f sample_cosine_weighted(const CGLA::Vec3f& normal)
{
  // ref: http://www.rorydriscoll.com/2009/01/07/better-sampling/
  // ref: http://pathtracing.wordpress.com/2011/03/03/cosine-weighted-hemisphere/
  // Get random numbers
  const float r1 = randomizer.mt_random();
  const float r2 = randomizer.mt_random();

  const float theta = acos(sqrt(1.0f - r1));
  const float phi = 2.0f * M_PI * r2;

	// Calculate new direction as if the z-axis were the normal
  CGLA::Vec3f _normal(sin(theta) * cos(phi), sin(theta) * sin(phi), cos(theta));

  // Rotate from z-axis to actual normal and return
  rotate_to_normal(normal, _normal);
  return _normal;
}
\end{lstlisting}

% ---------------------------------------------------- Exercise 5
\newpage
\section{Report Exercise 5}

Implemented global illumination with russian roulette and splitting at the first diffuse surface.\\
\\
Relevant figures: \ref{fig:globalillum}, and \ref{fig:cornellspherecaustics}. Relevant listings: \ref{lst:mclambertian::shade}, and \ref{lst:mclambertian::splitshade}.\\
\\
\begin{figure}[h]
	\centering
	\begin{subfigure}[b]{0.7\textwidth}
		\includegraphics[width=\textwidth]{week5/cornellblocks_75_164s_5spp_9rpp_split.png}
		\caption{Cornell box with blocks. 75 iterations, 164 seconds, 5 samples, 9 rays per pixel.}
		\label{fig:cornellblocksglobalillum}
	\end{subfigure}
	\begin{subfigure}[b]{0.7\textwidth}
		\includegraphics[width=\textwidth]{week5/cornellrightsphere_25_95s_5spp_9rpp.png}
		\caption{Cornell box with silver and glass sphere. 25 iterations, 95 seconds, 5 samples, 9rpp.}
		\label{fig:cornellspheresglobalillum}
	\end{subfigure}
	\caption{The Cornell box with diffuse, transparent, and metal materials. Both images use global illumination with splitting at the first diffuse surface and russian roulette for following diffuse surfaces.}
	\label{fig:globalillum}
\end{figure}

\begin{figure}[h]
	\centering
	\includegraphics[width=\textwidth]{week5/cornellrightsphere_25_95s_5spp_9rpp_caustics.png}
	\caption{Figure \ref{fig:cornellspheresglobalillum} with indicators for caustics.}
	\label{fig:cornellspherecaustics}
\end{figure}

In figure \ref{fig:cornellspherecaustics}, the caustics are marked with letters. The caustics next to letter $a$ are primary caustics, caused by light from the area light hitting the sphere on the top, and exiting it on the bottom (after refracting) and being reflected by the diffuse floor to the eye.
The caustics next to letter $c$ are hardly visible. They can be seen in the sphere's shadow best, being green caustics caused by reflected rays from the right wall. 
The caustics next to letter $b$ are caused by the area light being reflected from the silver sphere and transmitted through the glass sphere, arriving at the right wall.
\begin{lstlisting}[language=C++,caption=MCLambertian::split\_shade,label=lst:mclambertian::splitshade,firstnumber=73]
Vec3f MCLambertian::split_shade(Ray& r, bool emit) const
{
  Vec3f rho_d = get_diffuse(r);
  Vec3f result(0.0f);

  // indirect light
  for (unsigned int sample = 0; sample < samples; sample++)
  { 
    Ray hR; 
    tracer->trace_cosine_weighted(r, hR); // trace diffuse light in a hemisphere
    result += shade_new_ray(hR) * dot(r.hit_normal, hR.direction);
  }
     
  // f -> BRDF for Lambertian surfaces
  Vec3f f = rho_d / M_PI;
  result *= f / samples; // average sampled radiances

  result += Lambertian::shade(r, emit); // direct light

  return result;
}
\end{lstlisting}

\begin{lstlisting}[language=C++,caption=MCLambertian::shade,label=lst:mclambertian::shade,firstnumber=13]
Vec3f MCLambertian::shade(Ray& r, bool emit) const
{
  if(!r.did_hit_diffuse)
    return split_shade(r, emit);

  Vec3f rho_d = get_diffuse(r);
  double luminance = get_luminance(rho_d);
  Vec3f result(0.0f);
  
  float rand = mt_random();

  // the more light is being reflected by a surface, the higher the probability for reflection should be.
  // also: weight different diffuse colors depending on human eye sensitivity
  float y = 0.2989 * rho_d[0] + 0.5866 * rho_d[1] + 0.1145 * rho_d[2];
  //float y = (rho_d[0] + rho_d[1] + rho_d[2]) / 3.0f;
  
  // reference: http://www.youtube.com/watch?v=xIPKmbuVHQI
  if (rand < y)
  {
    // direct light
    result += Lambertian::shade(r, emit) / y; // divide by probability to make sure that monte carlo gives valid results

    // indirect light
    for (unsigned int sample = 0; sample < samples; sample++)
    { 
      Ray hR; 
      tracer->trace_cosine_weighted(r, hR); // trace diffuse light in a hemisphere
      result += shade_new_ray(hR) * dot(r.hit_normal, hR.direction) / y / samples;
    }
    
    // f -> BRDF for Lambertian surfaces
    Vec3f f = rho_d / M_PI;
    result *= f; // average sampled radiances
  }
  else // not tracing this ray any further
  {
    result += Lambertian::shade(r, emit);
  }

  return result;
}
\end{lstlisting}

\end{document}