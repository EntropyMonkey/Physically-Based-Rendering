 \section{Report Exercise 1}

\begin{itemize}
\item{Implemented a directional light with shadows}
\item{Implemented an area light with shadows}
\end{itemize}

Relevant pictures: figures \ref{fig:shadows}, and \ref{fig:arealight}.
Relevant listings: \ref{lst:directional}, \ref{lst:lambertian}, and \ref{lst:arealight}.

\begin{figure}
	\centering
	\begin{subfigure}[b]{0.4\textwidth}
		\includegraphics[width=\textwidth]{week1/bunny_noshadows.png}
		\caption{Without shadows: ~5s}
		\label{fig:bunnynoshadows}
	\end{subfigure}
	\begin{subfigure}[b]{0.4\textwidth}
		\includegraphics[width=\textwidth]{week1/bunny_shadowed.png}
		\caption{With shadows: ~8s}
		\label{fig:bunnyshadowed}
	\end{subfigure}
	\caption{Bunny.obj, Tris: 69451, 36 samples, 1 directional light, Lambertian shader}
	\label{fig:shadows}
\end{figure}

\begin{figure}
	\centering
	\begin{subfigure}[b]{0.4\textwidth}
		\includegraphics[width=\textwidth]{week1/arealight_noshadows.png}
		\caption{Without shadows: 0.3s}
		\label{fig:arealightnoshadows}
	\end{subfigure}
	\begin{subfigure}[b]{0.4\textwidth}
		\includegraphics[width=\textwidth]{week1/arealight_shadows.png}
		\caption{With shadows: 0.5s}
		\label{fig:arealightshadows}
	\end{subfigure}
	\caption{Cornellbox.obj and CornellBlocks.obj, Tris: 36, 4 samples, 1 area light, Lambertian shaders}
	\label{fig:arealight}
\end{figure}

\newpage
\begin{lstlisting}[language=C++,caption=Directional.cpp,label=lst:directional,firstnumber=15]
bool Directional::sample(const Vec3f& pos, Vec3f& dir, Vec3f& L) const
{
  dir = -light_dir;
  L = emission;

  // test for shadow
  Ray shadowRay(pos, -light_dir);
  bool inShadow = false;

  if (shadows)
    inShadow = tracer->trace(shadowRay);

  return !inShadow;
}
\end{lstlisting}

\begin{lstlisting}[language=C++,caption=Lambertian.cpp,label=lst:lambertian,firstnumber=16]
Vec3f Lambertian::shade(Ray& r, bool emit) const
{
  Vec3f rho_d = get_diffuse(r);
  Vec3f result(0.0f);
  
  // temp light direction and radiance
  Vec3f lightDirection, radiance;
  for (std::vector<Light*>::const_iterator it = lights.begin(); it != lights.end(); it++)
  {
    if ((*it)->sample(r.hit_pos, lightDirection, radiance))
    {
      // output of Lambertian BRDF
      Vec3f f = rho_d * M_1_PIf;
      
      // directional light radiance
      // f - scattered light radiance, radiance - current light radiance, last term: cosine cut off at 0
      result += f * radiance * std::max(dot(r.hit_normal, lightDirection), 0.0f);
    }
  }

  return result + Emission::shade(r, emit);
}
\end{lstlisting}

\begin{lstlisting}[language=C++,caption=AreaLight.cpp,label=lst:arealight, firstnumber=18]
bool AreaLight::sample(const Vec3f& pos, Vec3f& dir, Vec3f& L) const
{  
  // Get geometry info
  const IndexedFaceSet& geometry = mesh->geometry;
	const IndexedFaceSet& normals = mesh->normals;

  // averaged light position
  Vec3f lightPosition = Vec3f(0.0f);
  // averaged normals
  Vec3f lightNormal = Vec3f(0.0f);
  // emission summed up from all faces
  Vec3f emission = Vec3f(0.0f);

  // iterate over all faces
  for (int i = 0; i < geometry.no_faces(); i++)
  {
    // get the center of the face
    Vec3i face = geometry.face(i);
    Vec3f v0 = geometry.vertex(face[0]);
    Vec3f v1 = geometry.vertex(face[1]);
    Vec3f v2 = geometry.vertex(face[2]);
    Vec3f faceCenter = v0 + (v1 - v0 + v2 - v0) * 0.5f;
    
    // combine light position
    lightPosition += faceCenter;

    // average normals
    lightNormal += (normals.vertex(face[0]) + normals.vertex(face[1]) + normals.vertex(face[2])) / 3;

    // add emission
    emission += mesh->face_areas[i] * get_emission(i);
  }

  // average light position
  lightPosition /= geometry.no_faces();

  lightNormal.normalize();
  
  // get light direction and distance to light
  Vec3f lightDirection = lightPosition - pos;
  float lightDistance = length(lightDirection);

  // set area light direction, normalize
  dir = lightDirection / lightDistance;

  // set radiance
  L = emission * std::max(dot(-dir, lightNormal), 0.0f) / (lightDistance * lightDistance);

  // trace for shadows
  bool inShadow = false;
  if (shadows)
  {
    Ray shadowRay(pos, dir);
    shadowRay.tmax = lightDistance - 0.1111f;
    inShadow = tracer->trace(shadowRay);
  }

  return !inShadow;
}
\end{lstlisting}